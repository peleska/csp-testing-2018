% =========================================================================

\section{Introduction}
\label{sec:intro}

% =================================================================================

\subsubsection*{Motivation}

Model-based testing (MBT) is an  active research field that is currently
evaluated and integrated into industrial verification processes by many
companies. This holds particularly for the embedded and cyber-physical
systems domain. While MBT is applied in different flavours, we consider the
most effective variant to be the one where test cases and concrete test data,
as well as checkers for the expected results~(\emph{test oracles}), are
automatically generated from a reference model:~it guarantees the maximal
return of investment for the time and effort invested into creating the test
model. The test suites generated in this way, however, usually have different
test strength, depending on the generation algorithms applied.

For the safety-critical domain, test suites with guaranteed fault coverage
are of particular interest. For black-box testing, guarantees can be given
only if certain hypotheses are satisfied. These hypotheses are usually
specified by a \emph{fault domain}:~a set of models that may or may not
conform to the SUT. The so-called \emph{complete} test strategies guarantee
to uncover every conformance violation of the SUT with respect to a reference
model, provided that the true SUT behaviour is captured by a member of the
fault domain.

Generation methods for complete test suites have been developed for various
modelling formalisms. In this paper, we use \emph{Communicating Sequential
Processes (CSP)}~\cite{Hoare:1985:CSP:3921,Roscoe2010}; this is a mature
process-algebraic approach that has been shown to be well-suited for the
description of reactive control systems in many publications over almost five
decades. Industrial success has also been reported.

% ==================================================================================

\subsubsection*{Contributions}

This paper complements work published by two of the authors
in~\cite{DBLP:conf/pts/CavalcantiS17}. There, fault domains are specified as
collections of processes refining a  ``most general'' fault domain member.
With this concept of fault domains, complete test suites may be finite or
infinite. While this gives important insight into the theory of complete test
suites, we are particularly interested in finite suites when it comes to
their practical application.

Therefore, we present a complementary approach to the definition of CSP fault
domains in this paper. To this end, we observe that every finite-state CSP
process can be semantically represented as a finite normalised transition
graph, whose edges are labelled by the events the process engages in, and
whose nodes are labelled by minimal acceptances or, alternatively, maximal
refusals~\cite{Roscoe:1994:CME:197600}. The maximal refusals  express the
degree of nondeterminism that is present in a given CSP process state that is
in one-one-correspondence to a node of the normalised transition graph.
Inspired by the way that fault-domains are specified for finite state
machines (FSMs), we define them here as the set of CSP processes whose
normalised transition graphs do not exceed the size of the reference model's
graph by more than a give constant.

Our main contributions in this paper are as follows.
%
\begin{enumerate}
\item It is proven that for fault domains of the described type, complete
    test suite generation methods can be given for verifying (1)~trace
    equivalence, (2)~trace refinement, (3)~failures equivalence, and
    (4)~failures refinement.

\item We prove that finite complete test suites can be generated in all
    four cases, when using the fault domains based on the size of the
    members' normalised transition graphs.

\item We present test suite generation techniques for each of the four
    conformance relations by translating algorithms originally elaborated
    for the FSM domain into the CSP world. This translation preserves the
    completeness properties that have previously been established for the
    FSM domain by other authors.
\end{enumerate}
%
The possibility to translate complete test suites between different
formalisms (here FSMs and CSP processes) has been investigated before by two
of the authors~\cite{Huang2017}. \fxwarning{alcc: I miss a paragraph that
discusses what is difficult. So far, it sounds like just application of known
results, which is not true.}

% ==================================================================================

\subsubsection*{Overview}

@todo


% ==================================================================================
