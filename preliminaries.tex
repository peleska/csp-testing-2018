% =========================================================================
\section{Preliminaries}
\label{section:preliminaries}
% =========================================================================

In this section, we present the background material relevant to our work.

% =========================================================================
\subsection{Complete Test Suites}
\label{sec:fsmfm}
% =========================================================================

We use the term \emph{signature} to denote a collection of comparable models
represented in an arbitrary formalism. In this article, signatures represent
sets of finite state machines over fixed input and output alphabets, or CSP
processes with finite state, represented by their normalised transition
graphs (see Section~\ref{sec:ntg}).

Given a signature $Sig$  of models, a  \emph{fault model} ${\cal F} =
(M,\le,Dom)$ specifies a \emph{reference model} $M\in Sig$, a
\emph{conformance relation} $\le\ \subseteq Sig\times Sig$ between models,
and a \emph{fault domain} $Dom\subseteq Sig$. This terminology
follows~\cite{gotzhein_fault_1996}, where fault models have been originally
introduced in the context of testing of finite state machines. Note that
fault domains may contain both models conforming to the reference model and
models violating the conformance relation. Note further that the reference
model $M$ is not necessarily a member of the fault domain, although a model
of the SUT behaviour is. For example, $M$ could be nondeterministic, while
only deterministic implementation behaviours might be considered in the fault
domain.
%By $F(Sig,\le)$ we denote the set of all fault models ${\cal F}$
%defined for
%signature $Sig$ and conformance relation $\le$.

Let $\tc(Sig)$ denote the set of all \emph{test cases} applicable to elements
of $Sig$. This abstract notion of test case requires only the existence of a
relation $\pass\subseteq Sig\times \tc(Sig)$. For $(M,U)\in\pass$, the infix
notation $M\ \pass\ U$ is used, and interpreted as {\it `model $M$ passes the
test case $U$'}. If $(M,U)\not\in\pass$ holds, this is abbreviated by $M\
\tfail\ U$. Our specific notion of test cases for CSP models is elaborated in
Section~\ref{sec:csptc}.

A \emph{test suite} $\TS \subseteq \tc(Sig)$ denotes  a set of test cases. A
model $M$ \emph{passes the test suite} $\TS$, also written as $M\ \pass\
\TS$, if, and only if, $M\ \pass\ U$ for all $U\in \TS$. A test suite $\TS$
is called \emph{complete} for fault model ${\cal F} = (M,\le,Dom)$, if, and
only if, the following properties hold.
\begin{enumerate}
\item If a member $M'$ of the fault domain  conforms to the reference model $M$,
it passes the test suite, that is,
$$
\forall M'\in Dom: M'\le M \Rightarrow M'\ \pass\ \TS
$$
This property is usually called \emph{soundness} of the test suite.

\item If a member of the fault domain passes the test suite, it conforms to the reference model, that is,
$$
\forall M'\in Dom: M'\ \pass\ \TS \Rightarrow M'\le M
$$
This property is usually called \emph{exhaustiveness}.
\end{enumerate}
A test suite $\TS$ is \emph{finite} if it contains finitely many test cases and every test
case $U\in\TS$ is finite in the sense that it terminates after a finite number of steps.
It is trivial to see that, if $\TS$ is complete  for   ${\cal F} = (M,\le,Dom)$
and $Dom'\subseteq Dom$, then $\TS$ is also complete for ${\cal F}' = (M,\le,Dom')$.

% ============================================================================
\subsection{Translation of Test Cases and Test Suites}
\label{sec:transltt}

Let $Sig_1$ and $Sig_2$ be two signatures with conformance relations $\le_1$
and $\le_2$, and test case relations $\pass_1$ and $\pass_2$, respectively. A
function $T:\underline{Sig_1} \fun Sig_2$ defined on a sub-domain
$\underline{Sig_1} \subseteq Sig_1$ is called a \emph{model map}, and a
function $T^*:\tc(Sig_2) \fun \tc(Sig_1)$ is called a \emph{test case map}.
Note that models and test cases are mapped in opposite directions (see
Fig.~\ref{fig:satisfaction-relation}). Also, for $T$, we consider a
sub-domain of $Sig_1$ since it may be the case that the map is not defined
for the whole of $Sig_1$. For example, in our case, we are interested in
finite-state CSP processes only and their translation to graphs described in
the next section.

The pair $(T,T^*)$ fulfils the \emph{satisfaction condition} if, and only if,
the following conditions are fulfilled.
%
\begin{description}
  \item[\bf SC1] The model map is compatible with the conformance relations
      under consideration, in the sense that
  $$
  \forall {\cal S},{\cal S}'\in \underline{Sig_1}: {\cal S}' \le_1 {\cal S} \Leftrightarrow
  T({\cal S}') \le_2 T({\cal S}),
  $$
  so   the left-hand side diagram   in Fig.~\ref{fig:satisfaction-relation}
  commutes due to the fact that $T;\le_2 = \le_1;T$.\footnote{We note that
  the operator ``;'' denotes the relational composition defined for
  functions or relations $f\subseteq A\times B$, $g\subseteq B\times C$ by
  $f;g = \{(a,c)\in A\times C~|~\exists b\in B: (a,b)\in f \wedge (b,c)\in
  g\}$. Note that $f;g$ is evaluated from left to right (like composition
  of code fragments), as opposed to right-to-left evaluation which is
  usually denoted by $g\circ f$.}

  \item[\bf SC2] The model map and the test case map preserve the
      $\pass$-relationship in the sense that
  $$
  \forall {\cal S}\in \underline{Sig_1}, U \in\tc(Sig_2): T({\cal S})\ \pass_2\ U \Leftrightarrow {\cal S}\ \pass_1\ T^*(U),
  $$
  so the right-hand side diagram   in Fig.~\ref{fig:satisfaction-relation}
  commutes, due to the fact that $\pass_1 = T;\pass_2;T^*$.
\end{description}
% ============================================================================
The following theorem is a direct consequence
of~\cite[Theorem~2.1]{Huang2017}.
%
\begin{theorem}
  \label{th:theorytranslation}
  With the notation introduced above, let  $(T,T^*)$ fulfil the satisfaction condition.
  Suppose that $\TS_2 \subseteq TC(Sig_2)$ is a complete test suite
  for fault model ${\cal F}_2 = ({\cal S}_2,\le_2,Dom_2)$. Define fault model ${\cal F}_1$ on
  $\underline{Sig}_1$ by
  $$
  {\cal F}_1 = ({\cal S}_1,\le_1,Dom_1),\ \text{such that}\
  T({\cal S}_1)  =  {\cal S}_2\ \text{and}\
  Dom_1  =  \{ {\cal S}~|~T({\cal S})\in Dom_2 \}.
  $$
  Then
  $$
  \TS_1 = T^*(\TS_2)
  $$
  is a complete test suite with respect to fault model ${\cal F}_1$.
  \xbox
\end{theorem}
%
This theorem supports our justification of testing approaches for finite
state machines in the context of CSP.

% ............................................................................
\begin{figure}
\centering
\includegraphics[width=0.6\textwidth]{satisfaction-condition.pdf}
\vspace*{-20mm}
 \caption{Commuting diagrams reflecting the satisfaction condition.}
 \label{fig:satisfaction-relation}
 \end{figure}
% .............................................................................

% =========================================================================
\subsection{CSP and Refinement}

% =========================================================================

\subsubsection*{Communicating Sequential Processes} @todo
\fxwarning{alcc: I can make this small contribution.}

% =========================================================================
\subsubsection*{Normalised Transition Graphs for CSP Processes}
\label{sec:ntg}

As shown in~\cite{Roscoe:1994:CME:197600}, any finite-state CSP process $P$
can be represented by a \emph{normalised transition graph}
$$
G(P) = ( N, \ii n, \Sigma, t : N\times\Sigma \pfun N, r : N \fun \mathbb{P}\mathbb{P}(\Sigma)),
$$
with nodes $N$, initial node $\ii n\in N$, and process alphabet $\Sigma$. The
partial \emph{transition function} $t$ maps a node $n$ and an event
$e\in\Sigma$ to its successor node $t(n,e)$, if, and only if, $(n,e)$ is in
the domain of $t$, that is, there is a transition from $n$ with label $e$.
Normalisation of $G(P)$ is reflected by the fact that $t$ is a function.

A finite sequence of events $s\in\Sigma^*$ is a \emph{trace} of $P$, if there
is a path through $G(P)$ starting  at $\ii n$ whose edge labels coincide with
$s$. The set of traces of $P$ is denoted by $\trc(P)$. If $s\in\trc(P)$, then
the process corresponding to $P$ after having executed $s$ is denoted by
$P/s$. Since $G(P)$ is normalised, there is a unique node reached by applying
the events from $s$ one by one, starting in $\ii n$. Therefore, $G(P)/s$  is
also well-defined.

By $[n]^0$ we denote the \emph{fan-out} of $n$:~the set of events occurring
as labels in any outgoing transitions.
$$
[n]^0 = \{ e\in\Sigma~|~(n,e)\in\dom~t \}
$$
We also use this notation for CSP processes:~$[P]^0$ is the set
of events $P$ may engage into, in other words, the initials of $P$ after the
empty trace of events, that is, $initials(P/\langle\rangle)$ as defined
in~\cite{Roscoe2010}.

The total function $r$ maps each node $n$ to its \emph{refulsals} $r(n) =
\refs(n)$. Each element of $r(n)$ is a set of events that the CSP process $P$
might refuse to engage into, when in a process state corresponding to $n$.
The number of refusal sets in $\refs(P/s)$ specifies the degree of
nondeterminism that is present in process state $P/s$: the more refusal sets
contained in  $\refs(P/s)$, the more nondeterministic is the behaviour in
state $P/s$. If $P/s$ is deterministic, its refusals coincide with the set of
subsets of $\Sigma - [P/s]^0$, including the empty set.

For finite CSP processes, since the refusals of each process state are
subset-closed~\cite{Hoare:1985:CSP:3921,Roscoe2010}, $\refs(P/s)$ can be
re-constructed by knowing the set of \emph{maximal refusals}
$\maxrefs(P/s)\subseteq\refs(P/s)$. More formally, the maximal refusals
$\maxrefs(P/s)$ are defined as
$$
\maxrefs(P/s) = \{ R \in\refs(P/s)~|~\forall R'\in \refs(P/s) - \{ R\}: R \not\subseteq R'\}
$$
Conversely, with the maximal refusals at hand, we can reconstruct the refusals by subset-closure:
$$
\refs(P/s) = \{ R'\in\power(\Sigma)~|~\exists R\in \maxrefs(P/s): R'\subseteq R \}.
$$
To see that this approach works only for finite CSP processes, consider the
example where $\Sigma$ is infinite. In this case,
$\maxrefs(STOP/\langle\rangle)$ is empty, and so we cannot use this set to
calculate the refusals of $STOP$, that is, $\refs(STOP/\langle\rangle)$ as
defined above. As with refusals, we also use the transition graph-oriented
notation $\maxrefs(n) \subseteq r(n)$ to denote the maximal refusals
associated with graph state $n$: if $n$ is the state reached in the
transition graph by following the edge labels in trace $s$, then $\maxrefs(n)
= \maxrefs(P/s)$.

Well-formed normalised transition graphs must not refuse an event of the
fan-out of a state in {\it every} refusal applicable in this state; more
formally,
\begin{equation}
\label{eq:wellformedg}
\forall n\in N, e\in\Sigma: (n,e)\in\dom~t \Rightarrow
\exists R\in \maxrefs(n): e\not\in R
\end{equation}
%The total function $a$ maps each node to its set of \emph{minimal acceptances}:
%if $n\in N$ corresponds to a deterministic process state of $P$, $ac(n)$ contains a single acceptance $A\subseteq \Sigma$, and every $e\in A$ is in one-one-correspondence with a transition $t(n,e)$. If $n$ corresponds to a nondeterministic process state, $ac(n)$ contains at least two acceptances $A_1, A_2, \dots, A_k$. This reflects the fact that in a nondeterministic state, $P$ must accept all events of one acceptance
%$A_i, i \in \{ 1,\dots,k\}$, but may refuse all events $e$ from
%$A_j \setminus A_i, j\neq i$.
%
%Each well-defined transition graph $G(P)$ fulfils the following condition. The union of all minimal acceptances in each node corresponds to the set of events labelling its outgoing transitions.
%\begin{equation}
%\label{eq:wellformedg}
%\forall n\in N: (n,e)\in\dom~t \Leftrightarrow e\in\bigcup ac(n)
%\end{equation}
%In this condition, $\dom~t$ denotes the domain of function $t$.
By construction, normalised transition graphs reflect the \emph{failures
semantics} of finite-state CSP processes:~the traces $s$ of a process are
exactly the sequences of transition labels associated with paths through the
transition graph, starting at $\ii n$. The pairs $(s,R)$ with $s\in\trc(P)$
and $R\in r(G(P)/s)$ represent the failures of $P$.

% .....................................................................................
 \begin{figure}
 %%\hspace*{-40mm}
 \begin{center}
\includegraphics[width=\textwidth]{q0.pdf}
\end{center}
%%\vspace*{-10mm}
\caption{Normalised transition graph of CSP process $P$ from Example~\ref{ex:a}.}
 \label{fig:tga}
 \end{figure}
% .......................................................................................

\begin{example}{ex:a}
Consider the CSP process $P$ defined below.
\begin{eqnarray*}
P & = & a \then (Q\intchoice R)
\\
Q & = & a \then P \extchoice c \then P
\\
R & = & b \then P \extchoice c \then R
\end{eqnarray*}
Its transition graph $G(P)$ is shown in Fig.~\ref{fig:tga}. Process state $P$
is represented there as Node\_0, with $\{ b,c\}$ as the only maximal refusal,
since event $a$ can never be refused, and no other events are accepted.
Having engaged into $a$, the transition emanating from Node\_0 leads to
Node\_2 representing  the process state $P/a = Q\intchoice R$. The internal
choice operator induces several refusal sets derived from $Q$ and $R$. Since
these processes accept their initial events with external choice, process
$Q\intchoice R$ induces just two maximal refusal sets $\{b\}$ and $\{a\}$.
Note that event $c$ can never be refused, since it is not a member of any
maximal refusal.

Having engaged into $c$, the next process state is represented by Node\_1. Due to normalisation, there was only a single transition satisfying
$t(\text{Node\_2},c) = \text{Node\_1}$. This transition, however, can have been caused
by either $Q$ or $R$ engaging into $c$, so Node\_1 corresponds to process state
$Q/c \intchoice R/c = P \intchoice R$. This is reflected by the two maximal refusals
labelling Node\_1.
\end{example}

\subsubsection*{Tool Considerations}
The FDR tool~\cite{fdr} supports model checking and semantic analyses of CSP processes.
It provides an API~\cite{fdrmanual} that can be used to construct normalised transition graphs for CSP processes. The graph  nodes are labelled by \emph{minimal acceptances}. Since such a minimal acceptance set is the complement of a maximal refusal, the function $r$ mapping states
to their refusals can be implemented by creating the complements of all minimal acceptances
and then building all subsets of these complements. For practical applications,
the subset closure is never constructed in an explicit way; instead, sets are checked
with respect to containment in a maximal refusal.

%
%The maximal refusals in each process state $P/s$
% are the complements of
%the minimal acceptances of the node $n$
%corresponding to $P/s$. As a consequences, all failures
%of $P$ are represented by some $(s,R)$, where $s$ is an initialised path through the transition graph and $R\subseteq (\Sigma-A)$ for some minimal acceptance $A\in ac(n)$,
%such that $n$ is the node corresponding to $P/s$.




% =========================================================================
\subsection{Finite State Machines}


To make this paper sufficiently self-contained, we introduce definitions, notation, and facts
about
finite state machines (FSMs) that have been originally described in contributions on FSM testing, such as~\cite{petrenko_testing_2011,DBLP:conf/hase/PetrenkoY14,hierons_testing_2004}.

% -----------------------------------------------------------------------------------
A \emph{Finite State Machine (FSM)} is  a tuple
 $M=(Q, \ii{q}, \Sigma_I, \Sigma_O,  h)$   with state space $Q$, input alphabet $\Sigma_I$,
 output alphabet $\Sigma_O$, where $Q,\Sigma_I,\Sigma_O$ are finite and nonempty sets. $\ii{q}\in Q$ denotes the initial state.
$h\subseteq Q\times \Sigma_I \times \Sigma_O\times Q$ is the  transition relation,  $(q,x,y,q')\in h$ if and only if there is a transition from $q$ to $q'$ with input $x$ and output $y$.
We use  both set notation $(q,x,y,q')\in h$ and Boolean notation $h(q,x,y,q')$ for specifying
that $(q,x,y,q')$ is a transition in $h$.
We call $x$ a \emph{defined} input in state $q$, if there is a transition from $q$  with input $x$.
If every input of $\Sigma_I$ is defined in every state, $M$ is \emph{completely specified}.
If in every state $q$ and for every output $y\in\Sigma_O$, and input $x$ and a post-state
$q'$ satisfying $h(q,x,y,q')$ exists, the FSM is called \emph{output complete}.


FSM $M$ is called a \emph{deterministic FSM (DFSM)}, if for any state $q$ and defined input $x$,
$h(q,x,y,q') \wedge h(q,x,y',q'')$ implies $(y,q') = (y',q'')$. Intuitively speaking, a specific
input applied to a specific state uniquely determines both post-state and associated output.
If $M$ is not deterministic, it is called a \emph{nondeterministic FSM (NFSM)}.
If there is no emanating transition for $q\in Q$, this state is called a \emph{deadlock state}, and
\emph{$M$ terminates in $q$}. The set of deadlock states is denoted by $\deadlock(Q)\subseteq Q$.
The set of states that do not deadlock is denoted by
$\DF(Q) = \{ q\in Q~|~\exists (q',x,y,q'')\in h: q' = q \}$.


The transition relation $h$ can be extended in a natural way to input traces:
let $\overline{x}$ be an input trace and $\overline{y}$ an output trace. Then
$(q,\overline{x},\overline{y},q')\in h$, if and only if there is a transition sequence
from $q$ to $q'$ with input trace $\overline{x}$ and output trace $\overline{y}$.
If $q$ is the initial state $\ii{q}$, such a transition sequence is called an \emph{execution} of $M$. Executions are written in the notation
$$
q_0 \xrightarrow{x_1/y_1} q_1 \xrightarrow{x_2/y_2} \dots \xrightarrow{x_{k}/y_{k}} q_{k}
$$
with $q_0 = \ii{q}$, $h(q_{i-1},x_i,y_i,q_{i})$ for $i = 1,\dots,k$, and
$\overline{x} = x_1\dots x_k$ and $\overline{y} = y_1\dots y_k$.

The empty trace is denoted by $\varepsilon$, and
$(q,\varepsilon,\varepsilon,q)\in h$, for any state $q$.
A \emph{language}  of an FSM $M$  is the set consisting of all possible input/output
traces in $M$; we use notation
 $L_M(q)=\{\overline{x}/\overline{y}~|~\exists q'\in Q: h(q,\overline{x},\overline{y},q')\}$ for $q\in Q$, and  $L(M)=L_M(\ii{q})$.
By $\fsm(\Sigma_I,\Sigma_O)$ we denote the set of all FSMs with input alphabet $\Sigma_I$ and
output alphabet $\Sigma_O$.

An FSM $M$ is called \emph{observable} if in every state $q$, every existing post-state $q'$ is uniquely determined by the I/O pair $x/y$ satisfying $h(q,x,y,q')$. For
observable state machines, the partial function
$$
\_ \after\_ /\_ : Q\times\Sigma_I\times \Sigma_O \pfun Q;\
q\after(x/y) = q' \Leftrightarrow h(q,x,y,q')
$$
is well-defined. Again, it can be extended to I/O-traces $\overline x/\overline y$
by repetitive application.
Deterministic FSMs are always observable. Every non-observable FSM can be transformed into an observable one that has the same language~\cite{PeleskaHuangLectureNotesMBT}. The algorithm
operates in analogy to the algorithm normalising transition graphs of CSP processes.

% -----------------------------------------------------------------------------------
Two FSM $M_1, M_2$ are \emph{I/O-equivalent ($M_1\sim M_2$)} if and only if their languages coincide, i.e.~$L(M_1) = L(M_2)$. FSM $M_1$ is a \emph{reduction of $M_2$ ($M_1 \preceq M_2$)},
if and only if $L(M_1) \subseteq L(M_2)$. I/O-equivalence is also called
\emph{trace equivalence} by some authors, see, e.g.~\cite{luo_test_1994}.

% ------------------------------------------------------------------------------------


% -----------------------------------------------------------------------------------
FSMs can be composed in parallel by synchronising over common input/output events:
Let FSMs $M_i=(Q_i, \ii{q_i}, \Sigma_I, \Sigma_O,  h_i), i = 1,2$ be defined over
the same input/output alphabets. Then
$$
M_1 \cap M_2 = (Q_1\times Q_2, (\ii{q_1},\ii{q_2}),\Sigma_I, \Sigma_O, h)
$$
where the transition relation is specified by
$$
h((q_1,q_2),x,y,(q_1',q_2')) \Leftrightarrow h_1(q_1,x,y,q_1') \wedge h_2(q_2,x,y,q_2')
$$
By construction, $L(M_1 \cap M_2) = L(M_1) \cap L(M_2)$. Every execution
$$
(\ii{q_1},\ii{q_2}) \xrightarrow{x_1/y_1} (q_1^1,q_2^1)
\xrightarrow{x_2/y_2} \dots \xrightarrow{x_k/y_k} (q_1^{k},q_2^{k})
$$
of
$M_1\cap M_2$
is composed of executions
$$
\ii{q_1} \xrightarrow{x_1/y_1} q_1^1
\xrightarrow{x_2/y_2} \dots \xrightarrow{x_k/y_k} q_1^{k}\
\text{of}\ M_1\ \text{and}\
\ii{q_2} \xrightarrow{x_1/y_1} q_2^1
\xrightarrow{x_2/y_2} \dots \xrightarrow{x_k/y_k} q_2^{k}\
\text{of}\ M_2
$$
