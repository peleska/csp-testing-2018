
%%%%%%%%%%%%%%%%%%%%%%% file typeinst.tex %%%%%%%%%%%%%%%%%%%%%%%%%
%
% This is the LaTeX source for the instructions to authors using
% the LaTeX document class 'llncs.cls' for contributions to
% the Lecture Notes in Computer Sciences series.
% http://www.springer.com/lncs       Springer Heidelberg 2006/05/04
%
% It may be used as a template for your own input - copy it
% to a new file with a new name and use it as the basis
% for your article.
%
% NB: the document class 'llncs' has its own and detailed documentation, see
% ftp://ftp.springer.de/data/pubftp/pub/tex/latex/llncs/latex2e/llncsdoc.pdf
%
%%%%%%%%%%%%%%%%%%%%%%%%%%%%%%%%%%%%%%%%%%%%%%%%%%%%%%%%%%%%%%%%%%%

% =========================================================================

\documentclass[runningheads]{fac}

% =========================================================================

\usepackage[cmex10]{amsmath}
\usepackage{amssymb}
\setcounter{tocdepth}{3}
\usepackage{graphicx}

\usepackage{url}
\urldef{\mailsa}\path|{peleska,huang}@uni-bremen.de|
\urldef{\mailsb}\path|ana.cavalcanti@york.ac.uk|
\urldef{\mailsc}\path|adenilso@icmc.usp.br|

%\newcommand{\keywords}[1]{\par\addvspace\baselineskip
%\noindent\keywordname\enspace\ignorespaces#1}

\usepackage{float}
\usepackage{lscape}
\usepackage{lscape}

\usepackage{zed-csp}

\usepackage[draft]{fixme}

% =========================================================================

\DeclareMathSymbol{\B}{\mathalpha}{AMSb}{"42}
\DeclareMathSymbol{\I}{\mathalpha}{AMSb}{"49}
\DeclareMathSymbol{\N}{\mathalpha}{AMSb}{"4E}
\DeclareMathSymbol{\Pwr}{\mathalpha}{AMSb}{"50}
\DeclareMathSymbol{\Q}{\mathalpha}{AMSb}{"51}
\DeclareMathSymbol{\R}{\mathalpha}{AMSb}{"52}
\DeclareMathSymbol{\Z}{\mathalpha}{AMSb}{"5A}
\DeclareMathSymbol{\Sol}{\mathalpha}{AMSb}{"53}

% =========================================================================

\newcommand{\obsv}[1]{{\cal O}(#1)}
\newcommand{\ctrl}[1]{{\cal C}(#1)}
\newcommand{\vdot}[1]{\stackrel{.}{#1}}
%\newcommand{\power}{\mathbf{P}}

\newcommand{\HS}{{\mathcal H}}

\newcommand{\Interval}{\I}

\newcommand{\ist}{\mbox{{\tt true}}}
\newcommand{\isf}{\mbox{{\tt false}}}
\newcommand{\emptytrace}{\langle~\rangle}
\newcommand{\abegin}{\mathbf{begin}}
\newcommand{\aend}{\mathbf{end}}
\newcommand{\alet}{\mathbf{let}}
\newcommand{\aendlet}{\mathbf{endlet}}
\newcommand{\ain}{\mathbf{in}}
\newcommand{\afor}{\mathbf{for}}
\newcommand{\adownto}{\mathbf{downto}}
\newcommand{\aforall}{\mathbf{foreach}}
\newcommand{\awhile}{\mathbf{while}}
\newcommand{\ado}{\mathbf{do}}
\newcommand{\aod}{\mathbf{od}}
\newcommand{\aenddo}{\mathbf{enddo}}
\newcommand{\aif}{\mathbf{if}}
\newcommand{\afi}{\mathbf{fi}}
\newcommand{\athen}{\mathbf{then}}
\newcommand{\aelse}{\mathbf{else}}
\newcommand{\aelseif}{\mathbf{elseif}}
\newcommand{\aendif}{\mathbf{endif}}
\newcommand{\ainout}{\mathbf{inout}}
\newcommand{\awhere}{\mathbf{where}}
\newcommand{\areturn}{\mathbf{return}}
\newcommand{\aout}{\mathbf{out}}
\newcommand{\aprocedure}{\mathbf{procedure}}
\newcommand{\afunction}{\mathbf{function}}
\newcommand{\abreak}{\mathbf{break}}
\newcommand{\Sup}[1]{\overline{#1}}
\newcommand{\Inf}[1]{\underline{#1}}

\newcommand{\taba}{\hspace*{3mm}}
\newcommand{\tabb}{\hspace*{6mm}}
\newcommand{\tabc}{\hspace*{9mm}}
\newcommand{\tabd}{\hspace*{12mm}}
\newcommand{\tabe}{\hspace*{15mm}}
\newcommand{\tabf}{\hspace*{18mm}}

\newcommand{\gca}[1]{{{#1}^{\vartriangleright}}}
\newcommand{\gcb}[1]{{{#1}^{\vartriangleleft}}}
\newcommand{\gclr}{{{\vartriangleleft\atop\longleftarrow}\atop
                   {\longrightarrow\atop\vartriangleright}}}

\newcommand{\Nat}{{\mathbb N}}
\newcommand{\Real}{{\mathbb R}}

\newcommand{\trans}{\longrightarrow}
\newcommand{\transxxx}[1]{\stackrel{#1}{\longrightarrow}}

\newcommand{\transp}{\longrightarrow_{\power}}
\newcommand{\transl}{\longrightarrow_{L}}
\newcommand{\transg}{\longrightarrow_{G}}
\newcommand{\transcfg}[1]{\stackrel{#1}{\longrightarrow}_{CFG}}
\newcommand{\isdefd}{=_{\mbox{\footnotesize def}}}
\newcommand{\equivdef}{\equiv_{\mbox{\footnotesize def}}}
\newcommand{\mitem}{\mbox{\em M-Item}}
%\newcommand{\fun}{\rightarrow}
%\newcommand{\pfun}{\not\rightarrow}
\newcommand{\currt}{\hat{t}}

%\newcommand{\dom}{\mbox{dom}}
%\newcommand{\ran}{\text{ran}}

\newcommand{\sigmaa}{\sigma_A}
\newcommand{\strictimplies}{\stackrel{\bullet}{\Rightarrow}}

\newcommand{\trl}{/\!/}

\newcommand{\q}{\textbf{q}}

\newcommand{\eqc}[2]{[#1;#2]}

\newcommand{\vest}{V_{\text{\sl est}}}
\newcommand{\vmax}{V_{\text{\sl MRSP}}}
\newcommand{\wout}{\mathsf{W}}
\newcommand{\eout}{\mathsf{EB}}
\newcommand{\areb}{\mathsf{allowRevokeEB}}
\newcommand{\sbia}{\mathsf{SBAvailable}}
\newcommand{\sbz}{\mathsf{sb}_0}
\newcommand{\ticmd}{\mathsf{TICmd}}
\newcommand{\dmicmd}{\mathsf{DMICmd}}

\newcommand{\sob}{\mathsf{speedOnBoard}}
\newcommand{\std}{\mathsf{speedToDriver}}
\newcommand{\pstd}{\mathsf{permittedSpeedToDriver}}
\newcommand{\csmsw}{\mathsf{csmSwitch}}
\newcommand{\sbicmd}{\mathsf{sbiCmd}}
\newcommand{\sbidisplay}{\mathsf{DMIdisplaySBI}}

\newcommand{\dvw}{\mathsf{dV}_{\mathsf{warning}}}
\newcommand{\dvs}{\mathsf{dV}_{\mathsf{sbi}}}
\newcommand{\dve}{\mathsf{dV}_{\mathsf{ebi}}}

\newcommand{\calcw}{\mathsf{dV\_warning(float)}}
\newcommand{\calcs}{\mathsf{dV\_sbi(float)}}
\newcommand{\calce}{\mathsf{dV\_ebi(float)}}
\newcommand{\calcstd}{\mathsf{calc\_speed\_to\_driver()}}
\newcommand{\calcpstd}{\mathsf{calc\_permitted\_speed\_to\_driver()}}
\newcommand{\calcsob}{\mathsf{calc\_speed\_onboard()}}

\newcommand{\trc}{\text{traces}}
\newcommand{\failure}{{\text {\rm failures}}}
\newcommand{\qtrc}{\text{qtraces}}
\newcommand{\iotrc}{\text{L}}

%\newtheorem{definition}{Definition}
%\newtheorem{property}{Property}
%\newtheorem{lemma}{Lemma}
%\newtheorem{theorem}{Theorem}
%\newtheorem{corollary}{Corollary}
%\newtheorem{property}{Property}

%\newtheorem{note}{Note}[section]

\newcommand{\xbox}{\unskip\nobreak\hfil\penalty50
      \hskip2em\hbox{}\nobreak\hfil$\Box$%
      \parfillskip=0pt\finalhyphendemerits=0 \par%

      \medskip
      \parfillskip=0pt plus 1fil
}

\newcommand{\dontshow}[1]{}
\newcommand{\annot}[1]{\textbf{\textcolor{red}{#1}}}

\newcommand{\ta}{\mathbf{A}}
\newcommand{\te}{\mathbf{E}}
\newcommand{\tx}{\mathbf{X}}
\newcommand{\tf}{\mathbf{F}}
\newcommand{\tg}{\mathbf{G}}
\newcommand{\tu}{\mathbf{U}}
\newcommand{\tr}{\mathbf{R}}
\newcommand{\tw}{\mathbf{W}}

\newcommand{\ttu}[1]{\mathbf{U}^{#1}}
\newcommand{\ttg}[1]{\mathbf{G}^{#1}}
\newcommand{\ttf}[1]{\mathbf{F}^{#1}}

\newcommand{\fsm}{\text{FSM}}
\newcommand{\dfsm}{\text{DFSM}}
\newcommand{\nfsm}{\text{NFSM}}
\newcommand{\refmod}{\mathit{Ref}}

\newcommand{\refs}{\text{Ref}}
\newcommand{\maxrefs}{\text{maxRef}}

\newcommand{\passeq}{\underline{\text{pass}}_=}
\newcommand{\tfaileq}{\underline{\text{fail}}_=}

\newcommand{\passred}{\underline{\text{pass}}_\subseteq}
\newcommand{\tfailred}{\underline{\text{fail}}_\subseteq}

\newcommand{\pass}{\underline{\text{pass}}}
\newcommand{\tfail}{\underline{\text{fail}}}

\newcommand{\xpass}[1]{{#1}_\text{pass}}
\newcommand{\xfail}[1]{{#1}_\text{fail}}

\newcommand{\tc}{{\text{TC}}}
\newcommand{\TS}{{\text{TS}}}
\newcommand{\DL}{{\text{DL}}}
\newcommand{\deadlock}{{\it deadlock}}
\newcommand{\DF}{\text{DF}}

\newcommand{\ii}[1]{\underline{#1}}
\newcommand{\inc}{\,{\rm I}\,}

\newcommand{\scov}{\text{SCOV}}
\newcommand{\after}{\text{-after-}}
\newcommand{\reduction}{\preceq}

\newcommand{\ps}{\le_s}
\newcommand{\pss}[1]{\stackrel{#1}{\ps}}

\newcommand{\pe}{\sim_s}
\newcommand{\pes}[1]{\stackrel{#1}{\pe}}
\newcommand{\ol}{\overline}
\newcommand{\diag}{\mathbf{diag}}
\newcommand{\primem}{\mathbf{prime}}

\newtheorem{theorem}{Theorem}[section]
\newtheorem{lemma}{Lemma}[section]
\newtheorem{corollary}{Corollary}[section]

\newcounter{examplectr}
\newenvironment{example}[1]
{
{\refstepcounter{examplectr}

\medskip
\noindent
\bf Example~\theexamplectr.\label{#1}}
}
{
\unskip\nobreak\hfil\penalty50
      \hskip2em\hbox{}\nobreak\hfil$\Box$%
      \parfillskip=0pt \finalhyphendemerits=0 \par
}

% =========================================================================

\title{Finite Complete Suites for \\ CSP Refinement Testing}

% =========================================================================

\author{
Ana Cavalcanti,
Wen-ling Huang,
Jan Peleska,
and
Adenilso Simao
}

\correspond{Jan Peleska. University of Bremen,
           Department of Mathematics and Computer   Science.
            e-mail: peleska@uni-bremen.de, Tel: +49 421 218 63961, Fax: +49 421 218 3054}

\pubyear{xxxx}
\pagerange{\pageref{firstpage}--\pageref{lastpage}}

% ===========================================================================
\begin{document}
% =========================================================================

\maketitle

% =========================================================================

\begin{abstract}
  In this paper, new contributions to testing Communicating Sequential
Processes (CSP) are presented, with focus on the generation of complete,
finite test suites. A test suite is complete if it can uncover every
conformance violation of the system under test with respect to a reference
model. Both reference models and implementation behaviours are represented as
CSP processes. As conformance relation, we consider trace equivalence and
trace refinement, as well as failures equivalence and failures refinement.
Complete black-box test suites here rely on the fact that the SUT's true
behaviour is represented by a member of a fault-domain, that is, a collection
of CSP processes that may or may not conform to the reference model. We
define fault domains by bounding the number of excessive states occurring in
a fault domain member's representation as a normalised transition graph, when
comparing it to the number of states present in the graph of the reference
model. This notion of fault domains is quite close to the way they are
defined for finite state machines, and these fault domains guarantee the
existence of {\it finite} complete test suites.

  \keywords{model-based testing, complete testing theories, finite state
  machines}
\end{abstract}

% =========================================================================

\section{Introduction}
\label{sec:intro}

% =================================================================================

\subsubsection*{Motivation}

\subsubsection*{Main Contributions}


\subsubsection*{Overview}


% ==================================================================================



\section{Preliminaries}
% =========================================================================
\subsection{Complete Testing Theories}

\subsubsection*{Fault Models, Test Cases, Test Suites, and Completeness}
\label{sec:fsmfm}

We use the term \emph{signature} to denote a collection of comparable models represented 
in an arbitrary formalism. In this article, signatures represent sets of 
finite state machines
over fixed input and output alphabets, or CSP processes with finite state, represented 
by their normalised transition graphs.



Given a signature $Sig$  of models, a  \emph{fault model} ${\cal F} = (M,\le,Dom)$
specifies a \emph{reference model} $M\in Sig$, a \emph{conformance relation} 
$\le\ \subseteq Sig\times Sig$ between models, and a \emph{fault domain}
$Dom\subseteq Sig$. This terminology follows~\cite{gotzhein_fault_1996}, where
fault models were originally introduced in the context of finite state machine testing.
Note that fault domains may contain both models conforming to the reference model and
models violating the conformance relation. Note further that the reference model $M$ 
is not necessarily a member of the fault domain. For example, $M$ could be nondeterministic, while only deterministic implementation behaviours might be  considered in the fault domain.
By $F(Sig,\le)$ we denote the set of all fault models ${\cal F}$ 
defined for 
signature $Sig$ and conformance relation $\le$.

Let $\tc(Sig)$ denote the set of all \emph{test cases} applicable to
elements of $Sig$. The abstract notion of test cases defined here only requires the existence of  
 a  relation $\pass\subseteq Sig\times \tc(Sig)$. 
For
$(M,U)\in\pass$, the infix notation $M\ \pass\ U$ is used, and   interpreted as 
{\it `Model $M$ passes the test case $U$'}. 
If $(M,U)\not\in\pass$ holds, this is abbreviated by $M\ \tfail\ U$.




A \emph{test suite} $\TS \subseteq \tc(Sig)$ denotes  a set of test cases.
A model $M$ \emph{passes the test suite} $\TS$, also written as $M\ \pass\ \TS$,
if and only if $M\ \pass\ U$ for all $U\in \TS$. A test suite $\TS$ is called \emph{complete} for fault model ${\cal F} = (M,\le,Dom)$, if and only if the following properties hold.
\begin{enumerate}
\item If a member $M'$ of the fault domain  conforms to the reference model $M$, 
it passes the test suite, that is,
$$
\forall M'\in Dom: M'\le M \Rightarrow M'\ \pass\ \TS
$$
This property is usually called \emph{soundness} of the test suite.

\item If a member of the fault domain passes the test suite, it conforms to the reference model, that is,
$$
\forall M'\in Dom: M'\ \pass\ \TS \Rightarrow M'\le M
$$
This property is usually called \emph{exhaustiveness}.
\end{enumerate}
A test suite $\TS$ is \emph{finite} if it contains finitely many test cases and every test
case $U\in\TS$ is finite in the sense that it terminates after a finite number of steps.
It is trivial to see that, if $\TS$ is complete  for   ${\cal F} = (M,\le,Dom)$
and $Dom'\subseteq Dom$, then $\TS$ is also complete for ${\cal F}' = (M,\le,Dom')$.



%A \emph{complete testing theory} is a mapping (again denoted by $\TS$)
%$\TS : F \fun \Pwr(\tc(Sig))$
%from a set $F$ of fault models to test suites,  such that  the test suites
%$\TS({\cal F})$ are complete
%for all fault models  ${\cal F} \in F$. A testing theory 
%$\TS : F \fun \Pwr(\tc(Sig))$ is
%\emph{finite} if and only if 
%$\TS({\cal F})$ is a finite test suite for all ${\cal F}\in F$.
%In analogy, sound and exhaustive testing theories are defined.


% ============================================================================
\subsection{Translation of Testing Theories}
\label{sec:transltt}

Let $Sig_1$ and $Sig_2$ be two signatures with conformance relations $\le_1$ and $\le_2$,
and test case relations $\pass_1$ and $\pass_2$, respectively. 
A function $T:\underline{Sig_1} \fun Sig_2$ defined on a sub-domain $\underline{Sig_1} \subseteq Sig_1$  
 is called
a \emph{model map}, and a function $T^*:\tc(Sig_2) \fun \tc(Sig_1)$ is called a \emph{test case map}. Note that models and test cases are mapped in opposite directions 
(see Fig.~\ref{fig:satisfaction-relation}).
The pair $(T,T^*)$ fulfils the \emph{satisfaction condition} if and only if the following conditions {\bf SC1} and {\bf SC2} are fulfilled.
\begin{description}
\item[\bf SC1] The model map is compatible with the conformance relations under consideration, in the sense that
$$
\forall {\cal S},{\cal S}'\in \underline{Sig_1}: {\cal S}' \le_1 {\cal S} \Leftrightarrow 
T({\cal S}') \le_2 T({\cal S}),
$$
so   the left-hand side diagram   in Fig.~\ref{fig:satisfaction-relation} commutes due to the fact that
$T;\le_2 = \le_1;T$.\footnote{Operator ``;'' denotes the relational composition defined for
functions or relations $f\subseteq A\times B$, $g\subseteq B\times C$ by 
$f;g = \{(a,c)\in A\times C~|~\exists b\in B: (a,b)\in f \wedge (b,c)\in g\}$.
Note that $f;g$ is evaluated from left to right (like composition of code fragments), 
as opposed to right-to-left evaluation which is usually denoted by $g\circ f$.}

\item[\bf SC2] Model map and test case map preserve the $\pass$-relationship in the sense that
$$
\forall {\cal S}\in \underline{Sig_1}, U \in\tc(Sig_2): T({\cal S})\ \pass_2\ U \Leftrightarrow {\cal S}\ \pass_1\ T^*(U), 
$$
so the right-hand side diagram   in Fig.~\ref{fig:satisfaction-relation} commutes, due to the fact that
$\pass_1 = T;\pass_2;T^*$. 
\end{description}

% ............................................................................
\begin{figure}
\centering
\includegraphics[width=0.6\textwidth]{satisfaction-condition.pdf}
\vspace*{-20mm}
 \caption{Commuting diagrams reflecting the satisfaction condition.}
 \label{fig:satisfaction-relation}
 \end{figure}
% .............................................................................

 

% ============================================================================

 The following theorem is a direct consequence of~\cite[Theorem~2.1]{Huang2017}.

\begin{theorem}\label{th:theorytranslation}
With the notation introduced above, let  $(T,T^*)$ fulfil the satisfaction condition.
Suppose that $\TS_2 \subseteq TC(Sig_2)$ is a complete test suite
for fault model ${\cal F}_2 = ({\cal S}_2,\le_2,Dom_2)$. Define fault model ${\cal F}_1$ on 
$\underline{Sig}_1$ by
$$
{\cal F}_1 = ({\cal S}_1,\le_1,Dom_1),\ \text{such that}\
T({\cal S}_1)  =  {\cal S}_2\ \text{and}\
Dom_1  =  \{ {\cal S}~|~T({\cal S})\in Dom_2 \}.
$$
Then
$$
\TS_1 = T^*(\TS_2)
$$
is a complete test suite with respect to fault model ${\cal F}_1$.
\xbox
\end{theorem}

 
 


% =========================================================================
\subsection{CSP and Refinement}

% -------------------------------------------------------------------------
\subsubsection*{Normalised Transition Graphs}

As shown in~\cite{Roscoe:1994:CME:197600}, any finite-state CSP process $P$ can be represented by a \emph{normalised transition graph} 
$$
G(P) = ( N, \ii n, \Sigma, t : N\times\Sigma \pfun N, a : N \fun \mathbb{P}\mathbb{P}(\Sigma)),
$$
with nodes $N$, initial node $\ii n\in N$, and process alphabet $\Sigma$. The partial \emph{transition function} $t$ maps a node $n$ and an event $e\in\Sigma$ to its successor node $t(n,e)$, if and only if $(n,e)$ are in the domain of $t$. Normalisation of $G(P)$ is reflected 
by the fact that $t$ is a function. The total function $a$ maps each node to its set of \emph{minimal acceptances}: 
if $n\in N$ corresponds to a deterministic process state of $P$, $a(n)$ contains a single acceptance $A\subseteq \Sigma$, and every $e\in A$ is in one-one-correspondence with a transition $t(n,e)$. If $n$ corresponds to a nondeterministic process state, $a(n)$ contains at least two acceptances $A_1, A_2, \dots, A_k$. This reflects the fact that in a nondeterministic state, $P$ must accept all events of one acceptance 
$A_i, i \in \{ 1,\dots,k\}$, but may refuse all events $e$ from 
$A_j \setminus A_i, j\neq i$. 

Each well-defined transition graph $G(P)$ fulfils the following condition. The union of all minimal acceptances in each node corresponds to the set of events labelling its outgoing transitions.
\begin{equation}
\label{eq:wellformedg}
\forall n\in N: (n,e)\in\dom~t \Leftrightarrow e\in\bigcup a(n)
\end{equation}
In this condition, $\dom~t$ denotes the domain of function $t$. 

By construction, normalised transition graphs reflect the failures semantics of finite-state CSP processes: 
the traces $s$ of a process are exactly the paths through the transition graph, 
starting at $\ii n$. The maximal refusals in each process state $P/s$
 are the complements of 
the minimal acceptances of the node $n$ 
corresponding to $P/s$. As a consequences, all failures 
of $P$ are represented by some $(s,R)$, where $s$ is an initialised path through the transition graph and $R\subseteq (\Sigma-A)$ for some minimal acceptance $A\in a(n)$, 
such that $n$ is the node corresponding to $P/s$. 

\begin{example}\label{ex:a}
Consider CSP process 
\begin{eqnarray*}
P & = & a \then (Q\intchoice R)
\\
Q & = & a \then P \extchoice c \then P
\\
R & = & b \then P \extchoice c \then R
\end{eqnarray*}
Its transition graph $G(P)$ is shown in Fig.~\ref{fig:tga}. Process state $P$ is represented there as Node\_0, with $\{ a\}$ as the only acceptance, since event $a$ can never be refused, and no other events are accepted. Having engaged into $a$, the transition emanating from Node\_0 leads to Node\_2 representing  the process state 
$P/a = Q\intchoice R$. The internal choice operator induces several acceptance sets derived from $Q$ and $R$. Since these processes accept their initial events with external choice, 
process $Q\intchoice R$ induces just two minimal acceptance sets $\{a,c\} = [Q]^0$ and
$\{b,c\} = [R]^0$. Note that event $c$ can never be refused, since it is a member of all minimal acceptances. 

Having engaged into $c$, the next process state is represented by Node\_1. Due to normalisation, there was only a single transition satisfying 
$t(\text{Node\_2},c) = \text{Node\_1}$. This transition, however, can have been caused 
by either $Q$ or $R$ engaging into $c$, so Node\_1 corresponds to process state
$Q/c \intchoice R/c = P \intchoice R$. This is reflected by the two minimal acceptances
labelling Node\_1. 
\xbox
\end{example}


% .....................................................................................
 \begin{figure}
 %%\hspace*{-40mm}
 \begin{center}
\includegraphics[width=.5\textwidth]{q0.pdf}
\end{center}
%%\vspace*{-10mm}
\caption{Normalised transition graph of CSP process $P$ from Example~\ref{ex:a}.}
 \label{fig:tga}
 \end{figure}
% ....................................................................................... 



% =========================================================================
\subsection{Finite State Machines}

\section{Finite Complete Testing Theories for CSP}
\label{sec:finitecomplete}
% ==========================================================================
\subsection{A Model Map from CSP Processes to Finite State Machines}
\label{sec:mmap}

We will now construct a model map for associating CSP processes represented by normalised transition graphs to finite state machines. The intuition behind this 
construction is that the finite state machine's input alphabet  corresponds to
{\it sets of inputs} that may be offered to a CSP process. Depending on the events 
contained in this set, the process may (1) accept all of them, (2) accept some of them while refusing others, and (3) refuse all of them. This is reflected in the FSM
by output events that represent events that the process really has engaged in and 
an extra event $\bot$ representing deadlock, if the set of events has been refused.

More formally, we fix a finite CSP process alphabet $\Sigma$ and consider a 
finite-state process 
$P$ over this alphabet with normalised transition graph 
$G(P)=( N, \ii n, \Sigma, t : N\times\Sigma \pfun N, ac : N \fun \mathbb{P}\mathbb{P}(\Sigma))$,
 then the model map $T$ maps $P$ to the following observable FSM $T(P) = (Q,\ii q, \Sigma_I,\Sigma_O,h)$ satisfying
\begin{eqnarray*}
Q & = & N\cup \{\DL\} 
\\
\ii q & = & \ii n
\\
\Sigma_I & = & \power(\Sigma) - \{ \varnothing \}
\\
\Sigma_O & = & \Sigma \cup \{ \bot\}
\\
h & = & \{ (n,A,e,n')~|~A\in \Sigma_I \wedge e\in A \wedge 
(n,e)\in \dom~t\wedge t(n,e) = n' \} \cup {}
\\ & & 
\{ (n,A,\bot,\DL)~|~A\in \Sigma_I   \wedge 
\exists A'\in ac(n) \wedge A\cap A' = \varnothing
  \}
\end{eqnarray*} 

\begin{example}\label{ex:b}
For the CSP process $P$ and its transition graph $G(P)$ discussed in Example~\ref{ex:a}, the FSM $T(P)$ is depicted in Fig.~\ref{fig:fsm0}. 
For displaying its transitions, we used notation
$$
\forall A: \text{condition} / e
$$
which stands for a set of transitions between the respective nodes: one transition per non-empty set $A\subseteq \Sigma$ fulfilling the specified condition.
The arrow Node\_0 $\longrightarrow$ Node\_2 labelled by $\forall A: a\in A / a$, 
for example, stands for FSM transitions
$$
\begin{array}{l}
\text{Node\_0} \xrightarrow{\{a\}/a} \text{Node\_2} \\
\text{Node\_0} \xrightarrow{\{a,b\}/a} \text{Node\_2} \\
\text{Node\_0} \xrightarrow{\{a,c\}/a} \text{Node\_2} \\
\text{Node\_0} \xrightarrow{\{a,b,c\}/a} \text{Node\_2} \\
\end{array}
$$
\xbox
\end{example}


% ...................................................................................
 \begin{figure}
 %%\hspace*{-40mm}
 \begin{center}
\includegraphics[width=\textwidth]{fsm0.pdf}
\end{center}
%%\vspace*{-10mm}
\caption{FSM resulting from applying the model map to CSP process $P$ from Example~\ref{ex:a}.}
 \label{fig:fsm0}
 \end{figure}
% ................................................................................... 

We are now in the position to state and prove the theorem about the model map 
fulfilling the satisfaction condition {\bf SC1} introduced in Section~\ref{sec:transltt}. To this end, we first introduce three lemmas.



\begin{lemma}
Let $s\in \Sigma^*$ be any trace of $P$. Let $\underline n$ be the initial node of $G(P)$ and the initial state of $T(P)$. Let $n$ be the node of $G(P)$ denoting the process state $P/s$. Then for any input sequences $x\in \Sigma_I^*$ satisfying 
$\#x = \#s = k$   and $s(i)\in x(i)$, $\forall i\le k$, we have $x/s\in L(T(P))$ and $\underline n{\text{-after-}}x/s=n$. Furthermore,
for any $B\in \Sigma_I$, 
$$
B\in {\text{\rm Ref}}(P/s)\Leftrightarrow (x/s).(B/\bot) \in L(T(P))
$$

Conversely, any $x/s\in L(T(P))$ fulfils 
\begin{itemize}
\item[Case (1)] $s\in {\text {\rm tr}}(P)$ and $s(i)\in x(i), \forall i\le \#(x/s)$ \item[Case (2)] $x/s=(x'.x_k)/(s'.\bot)$ with $s'\in {\text {\rm tr}}(P)$, $s(i)\in x(i), \forall i< \#(x/s)$ and $x_k\in {\text{\rm Ref}}(P/s')$. 
\end{itemize}
\end{lemma}


From above we have straightforward the following lemmas
%----------------------------------------------------------------------------------
\begin{lemma}
For any $s\in \Sigma^*$, 
$$s\in {\text {\rm tr}}(P) \Leftrightarrow \exists x\in \Sigma_I^*: x/s \in L(T(P))$$
\end{lemma}
%-----------------------------------------------------------------------------

%--------------------------------------------------------------------------------
\begin{lemma}
For any $s\in \Sigma^*$ and $x_k\in \Sigma_I$, 
$$s\in {\text {\rm tr}}(P) \wedge  x_k\in {\text {\rm Ref}}(P/s)\Leftrightarrow \exists x\in \Sigma_I^*: x.x_k/s.\bot \in L(T(P))$$
\end{lemma}
%-------------------------------------------------------------------------------

\begin{theorem}
\label{th:sc1}
Consider the signature $Sig_1$ of CSP processes over fixed alphabet $\Sigma$ and the 
model map $T$ from CSP processes to finite state machines specified above.
Then
$$
\forall P, Q\in Sig_1: P \lessdet_F Q \Leftrightarrow   T(Q) \preceq T(P),
$$
where $\lessdet_F$ denotes failures refinement and $\preceq$ denotes reduction.
\end{theorem}
\begin{proof}
\[\begin{array}{lll}
T(Q)\le T(P) &\Leftrightarrow & L(T(Q))\subseteq L(T(P))\\
&\Leftrightarrow & \forall x/s\in L(T(Q))\Rightarrow x/s\in  L(T(P))\\
&\Leftrightarrow & \big(\forall x/s\in L(T(Q))\wedge s\in \Sigma^*\Rightarrow x/s\in  L(T(P))\wedge s\in \Sigma^* \big)\wedge\\
& &  \big(\forall x.x_k/s.\bot\in L(T(Q))\wedge s\in\Sigma^*\Rightarrow x.x_k/s.\bot\in  L(T(P))\wedge s\in\Sigma^*\big) \\
&\Leftrightarrow & \big(\forall s\in {\text {\rm tr}}(Q) \wedge \forall x/s\in L(T(Q))
 \Rightarrow  s\in {\text {\rm tr}}(P) \wedge   x/s\in L(T(P))\big)\wedge \\
& & \big(\forall s\in {\text {\rm tr}}(Q) \wedge\forall x/s\in L(T(Q))\wedge \forall x_k\in {\text {\rm Ref}}(Q/s)\setminus\{\varnothing\}\\&&{\phantom{\big(}}\Rightarrow  s\in {\text {\rm tr}}(P)\wedge  x/s\in L(T(P))\wedge  x_k\in {\text {\rm Ref}}(P/s)\setminus\{\varnothing\}\big) \\

&\Leftrightarrow & \big(\forall s\in {\text {\rm tr}}(Q) \Rightarrow  s\in {\text {\rm tr}}(P) \big) \wedge \\
& & \big(\forall s\in {\text {\rm tr}}(Q) \wedge \forall x_k\in {\text {\rm Ref}}(Q/s)\Rightarrow  s\in {\text {\rm tr}}(P) \wedge  x_k\in {\text {\rm Ref}}(P/s)\big) \\
&\Leftrightarrow &  \forall s\in {\text {\rm tr}}(Q)\Rightarrow s\in{\text {\rm tr}}(P) \wedge {\text {\rm Ref}}(Q/s)\subseteq {\text {\rm Ref}}(P/s)\\
&\Leftrightarrow & P\,\sqsubseteq_F\, Q

\end{array}
\]
\end{proof}
 

% ==========================================================================
\subsection{A Test Case Map from Finite State Machines to CSP Processes}
\label{sec:tcmap}

% -------------------------------------------------------------------------
\subsubsection*{FSM Test Cases}

Following~\cite{DBLP:conf/hase/PetrenkoY14}, 
an \emph{adaptive FSM test case} 
$$
tc_\text{FSM}=(Q,\ii q,\Sigma_I,\Sigma_O,h,in)
$$ 
is a nondeterministic, observable, output-complete, acyclic FSM which only provides a single input in a given state. Running in FSM intersection mode with the SUT, the test case provides a specific input to the SUT; this input is determined by the current state of the test case. It accepts every output and transits either to a fail-state FAIL, if the output is wrong according to the test objectives, or to the next test state uniquely determined  by the processed input/output pair. Another state PASS indicates that
the test has been completed without failure. Both FAIL and PASS are termination states, that is, they do not have any outgoing transitions.

Since the test case state determines the input for all of its outgoing transitions, this input is typically used as a state label, and the outgoing transitions are just labelled by the possible outputs. A function $in : Q -\{  \text{PASS}, \text{FAIL} \} 
\fun\Sigma_I$ maps the states to these inputs. Termination states
of the FSM are not labelled with further inputs.

\begin{example}
Consider the FSM test case depicted in Fig.~\ref{fig:fsm0tc} which is specified
for the same input and output alphabets as defined for  the FSM presented in Example~\ref{ex:b}. The test case is passed by the FSM from Example~\ref{ex:b}, because 
intersecting the two state machines results in an FSM which always reaches the PASS state.
\xbox
\end{example}


% ...................................................................................
 \begin{figure}
 %%\hspace*{-40mm}
 \begin{center}
\includegraphics[width=.8\textwidth]{fsm0tc.pdf}
\end{center}
%%\vspace*{-10mm}
\caption{An FSM test case which is passed by the FSM presented in Example~\ref{ex:b}.}
 \label{fig:fsm0tc}
 \end{figure}
% ................................................................................... 

 


% -------------------------------------------------------------------------
\subsubsection*{CSP Test Cases}
A \emph{CSP test case} is a terminating process with alphabet 
$\Sigma\cup\{\dag,\bot,\tick \}$, where the extra events stand for 
(1) test  verdict FAIL ($\dag$), (2) timeout ($\bot$), and (3) test 
 verdict PASS ($\tick$). In principle, very general classes of CSP processes can be
 used for testing, as introduced, for example, in~\cite{peleska_testing_1996,peleska1997a}. For the purpose of this paper, however, we can restrict the possible variants of CSP test cases to the ones that are in the range of the test case map which is constructed next.



% -------------------------------------------------------------------------
\subsubsection*{Test Case Map}

The test case map $T^* :TC(FSM) \fun TC(CSP)$ is specified with respect to a fixed
CSP process alphabet $\Sigma$ extended by the events $\{\dag,\bot,\tick \}$ introduced
above
and the associated FSM input and output alphabets
$\Sigma_I = \power(\Sigma)-\{\varnothing\}$ and $\Sigma_O=\Sigma\cup \{\bot \}$.
Given an FSM test case $tc_\text{FSM}=(Q,\ii q,\Sigma_I,\Sigma_O,h,in)$, 
the image $T^*(tc_\text{FSM})$ is the CSP process $tc_\text{CSP}$ specified 
as follows.
\begin{eqnarray*}
tc_\text{CSP} & = & tc(\ii q)
\\
tc(q) & = & \big( e : \{ a\in in(q)~|~h_1(q,in(q),e)\notin \{\text{PASS},\text{FAIL}\} \} \bullet e \then tc(h_1(q,in(q),e) \big)
\\ & & \extchoice
\\ & & \big( e : \{ a\in in(q)~|~h_1(q,in(q),e) = \text{PASS} \} \bullet e \then \tick \then \Skip \big)
\\ & & \extchoice
\\ & & \big( e : \{ a\in in(q)~|~h_1(q,in(q),e) = \text{FAIL} \} \bullet e \then \dag \then \Skip \big)
\end{eqnarray*}


% ..........................................................................
\begin{example}
The FSM test case $tc_\text{FSM}$
shown in Fig.~\ref{fig:fsm0tc} is mapped by $T^*$ to the following
CSP test case.
\begin{eqnarray*}
T^*(tc_\text{FSM}) & = & P_1
\\
P_1 & = & \big(e:\{ b,c,\bot \} \bullet e \then \dag\then\Skip\big)
\extchoice
 \big( a \then P_2 \big) 
\\
P_2 & = & \big( e : \{ a,b,\bot \}\bullet e\then \dag\then\Skip  \big)
\extchoice
\big( c\then P_3 \big)
\\
P_3 & = & \big( e : \{ a,b \}\bullet e\then \dag\then\Skip  \big)
\extchoice
\big( \bot \then P_4 \big)
\extchoice
\big( c\then P_5 \big)
\\
P_4 & = & \big( e : \{ b,c,\bot \}\bullet e\then \dag\then\Skip  \big)
\extchoice
\big( a\then\tick\then \Skip \big)
\\
P_5 & = & \big( e : \{ a,b,\bot \}\bullet e\then \dag\then\Skip  \big)
\extchoice
\big( c\then\tick\then \Skip \big)
\end{eqnarray*}
\xbox
\end{example}


% ==========================================================================




\section{Case Studies}
\label{sec:case}
% =======================================================================
 
 


% ====================================================================== 

\input{related}

\input{conclusion}

% =====================================================================================

\bibliographystyle{alpha}
\bibliography{references,jp}

% =====================================================================================
\end{document}
% =========================================================================
